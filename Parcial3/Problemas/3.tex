\section{Problema 3}
Sea $Y_{1}, Y_{2}, \ldots, Y_{n}$ una muestra aleatoria de tamaño $n$ de una población con distribución uniforme en el intervalo $(0, \theta) .$ Sean $Y_{(n)}=\max\left(Y_{1}, Y_{2}, \ldots, Y_{n}\right)$ y $U=\frac{1}{\theta} Y_{(n)}$.
\begin{enumerate}
\item a) Encuentre la función de distribución acumulada $F_{U}(u)$.
\begin{tcolorbox}[colback=gray!15,colframe=purple!10!purple,title=Sección 6.7 - Estadísticos de Orden (Order Statistics)]
   Considerando las variables ordenadas aleatorias $Y_1,Y_2,...,Y_n$ donde $Y_{(1)}\leq Y_{(2)}\leq\cdots\leq Y_{(n)}$. La notación propuesta por el libro: 
   $$Y_{(1)}=\min (Y_1,Y_2,\cdots, Y_n)$$
   $$Y_{(n)}=\max (Y_1,Y_2,\cdots, Y_n)$$
   \textbf{Resumiendo}
   \begin{enumerate}
       \item $g_{(n)}(y)$ denota la función densidad de $Y_{(n)}$, entonces:
   $$g_{(n)}(y)=n\left[F(y)\right]^{n-1}f(y)$$
        \item La función distribución de $Y_{(n)}$ es dado por: 
        $$F_{Y_{(n)}}=P(Y_{(n)}\leq y)=P(Y_1\leq y)P(Y_2\leq y)\cdots P(Y_n\leq y)=[F(y)]^n$$
   \end{enumerate}
    \end{tcolorbox}
\begin{tcolorbox}[colback=gray!15,colframe=blue!10!blue,title=Definición 4.6 - Distribución Uniforme-Densidad]
$$f(y)= \begin{cases}\frac{1}{\theta_2-\theta_1}, & \theta_1\leq y\leq \theta_2\\ 0, &\text{ cualquier otro}\end{cases}$$
    \end{tcolorbox}
    

\begin{solution}
La función densidad de la distribución uniforme en el intervalo $(0,\theta)$ es: 
$$f(y)= \begin{cases}\frac{1}{\theta}, & 0\leq y\leq \theta\\ 0, &\text{ cualquier otro}\end{cases}$$
Es decir que integrando la función densidad tenemos la función distribución: 
$$F(y)= \begin{cases}\frac{y}{\theta}, & 0\leq y\leq \theta\\ 0, &\text{ cualquier otro}\end{cases}$$
Comenzamos proponiendo el enunciado de la \textbf{Sección 6.7} que dice que:
\begin{align*}
    F_{Y_{(n)}}(y) & = [F(y)]^n\\
                &= \left[\frac{y}{\theta}\right]^n, \quad 0\leq y\leq \theta 
\end{align*}
Ahora, por otra parte, calculando la función distribución de $U$, tenemos (usando la definción usual): 
\begin{align*}
    F_U(u) &= P(U\leq u)\\
           &= P\left(\frac{Y_{(n)}}{\theta}\leq u\right)\\
           &= P\left(Y_{(n)}\leq \theta u\right)\\
           &= F_{Y_{(n)}}(\theta u) 
\end{align*}
en donde es trivial ver (analizando sus casos) que: 
$$F_U(u)=\begin{cases}0, & \theta u\leq0\\
\left(\frac{\theta u}{\theta}\right)^n, & 0\leq \theta u\leq \theta\\
1, & \theta u \geq 1
\end{cases}$$

Aquí, haremos una suposición para un mejor manejo del problema, como en el \textbf{Ejemplo 8,5} del libro de texto, en donde se puede asumir que $U$ es uniformemente distribuido sobre $[0,1]$. Por lo tanto: 

$$F_U(u)=\begin{cases}0, &  u\leq0\\
\left(u\right)^n, & 0\leq u\leq 1\\
1, &  u \geq 1
\end{cases}$$
    
    \end{solution}
\item b) Demuestre que $\frac{1}{\theta} Y_{(n)}$ es una cantidad pivote.
\begin{tcolorbox}[colback=gray!15,colframe=blue!10!blue,title=Sección 8.5-Intervalos de confianza - Método del Pivote]
El método depende de encontrar una cantidad pivote que posee dos características: 
\begin{enumerate}
    \item Es una función de las medidas de la muestra y el parámetro desconocido $\theta$, donde $\theta$ es la única cantidad desconocida. 
    \item Su función distribución no depende sobre el parámetro $\theta$.
\end{enumerate}
    \end{tcolorbox}
\begin{solution}
   Entonces, nos están pidiendo (asumiendo $h$ como el intervalo de confianza al que debemos llegar): 
   \begin{align*}
       P(U\leq a) &=  P\left(\frac{Y_{(n)}}{\theta}\leq a\right)\\
       &= F_U(a)\\
       &=h
   \end{align*}
   Entonces $a^n=h$, tal que $a=(h)^{1/n}$. Por lo tanto, $U=\frac{Y_{(n)}}{\theta}$ es una cantidad pivote; ya que es independiente. 
   
    \end{solution}
\item c) Use la cantidad de pivote del inciso b) para hallar un límite de confianza inferior de $95 \%$ para $\theta\left(\mathrm{P}\left(\hat{\theta}_{L} \leq \theta\right)=1-\alpha\right)$
\begin{solution}
    Usando el inciso anterior, en donde, $h=0.95$. Tenemos que $a=(0.95)^{1/n}$. Por lo tanto, podemos asumir lo siguiente:
    
    $$
\begin{aligned}
0.95 &=P\left(U \leq 0.95^{\frac{1}{n}}\right) \\
&=P\left(\frac{Y_{(n)}}{\theta} \leq 0.95^{\frac{1}{n}}\right) \\
&=P\left(\theta \geq \frac{Y_{(n)}}{0.95^{\frac{1}{n}}}\right)
\end{aligned}
$$
Concluimos que el límite de confianza de  $95 \%$ para $\theta$ está definido como:
$$
\theta \in\left\lfloor\frac{Y_{(n)}}{0.95^{\frac{1}{n}}}, \infty\right)
$$
    \end{solution}
\end{enumerate}
(Valor 25 puntos).
