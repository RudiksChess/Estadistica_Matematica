\documentclass[a4paper,12pt]{article}
\usepackage[top = 2.5cm, bottom = 2.5cm, left = 2.5cm, right = 2.5cm]{geometry}
\usepackage[T1]{fontenc}
\usepackage[utf8]{inputenc}
\usepackage{multirow} 
\usepackage{booktabs} 
\usepackage{graphicx}
\usepackage[spanish]{babel}
\usepackage{setspace}
\setlength{\parindent}{0in}
\usepackage{float}
\usepackage{fancyhdr}
\usepackage{amsmath}
\usepackage{amssymb}
\usepackage{amsthm}
\usepackage{natbib}
\usepackage{graphicx}
\usepackage{subcaption}
\usepackage{booktabs}
\usepackage{etoolbox}
\usepackage{apalike}
\usepackage{minibox}
\usepackage{hyperref}
\usepackage{xcolor}
\usepackage{tcolorbox}
\newcommand{\linea}{\noindent\rule{\textwidth}{1pt}}
\AtBeginEnvironment{align}{\setcounter{equation}{0}}
\newenvironment{solution}
  {\renewcommand\qedsymbol{$\square$}\begin{proof}[\textcolor{blue}{Solución}]}
  {\end{proof}}

\pagestyle{fancy}

\fancyhf{}

\lhead{\footnotesize Estadística Matemática}
\rhead{\footnotesize  Rudik Roberto Rompich}
\cfoot{\footnotesize \thepage}

\begin{document}
    \thispagestyle{empty} 
    \begin{tabular}{p{15.5cm}}
    \begin{tabbing}
    \textbf{Universidad del Valle de Guatemala} \\
    Departamento de Matemática\\
    Licenciatura en Matemática Aplicada\\\\
   \textbf{Estudiante:} Rudik Roberto Rompich\\
   \textbf{E-mail:} \textcolor{blue}{ \href{mailto:rom19857@uvg.edu.gt}{rom19857@uvg.edu.gt}}\\
   \textbf{Carné:} 19857
    \end{tabbing}
    \begin{center}
        MM2036 - Estadística Matemática - Catedrático: Paulo Mejía\\
        \today
    \end{center}\\
    \hline
    \\
    \end{tabular} 
    \vspace*{0.3cm} 
    \begin{center} 
    {\Large \bf Parcial 4
} 
        \vspace{2mm}
    \end{center}
    \vspace{0.4cm}
%---------------------------
%\begin{tcolorbox}[colback=gray!15,colframe=black!1!black,title=A nice heading]
%\end{tcolorbox}

%\fbox{lol}
%---------------------------


%-----------------------------
Instrucciones: Resuelva los siguientes problema. Favor hacer la solución en latex y cargar el archivo latex y pdf en la tarea de Canvas. Para la resolución de los problemas, se utilizará el libro de \cite{wackerly2014mathematical}. 
\section{Problema 1}


Sea $Y_{1}, Y_{2}, \ldots, Y_{n}$ una muestra aleatoria de una distribución normal con media $\mu$ y varianza $1 .$
\begin{tcolorbox}[colback=gray!15,colframe=black!1!black,title=Definición 4.8 - Distribución Normal]
	A random variable $Y$ is said to have a normal probability distribution if and only if, for $\sigma >0$ and $-\infty<\mu<\infty$, the density function of $Y$ is
	$$f(y)=\frac{1}{\sigma\sqrt{2\pi}}e^{-(y-\mu)^2/(2\sigma^2)}, \qquad \infty<y<\infty$$
	\end{tcolorbox}
\begin{enumerate}
	\item a) Demuestre que $\overline{Y}$ es un estimador suficiente para $\mu$.
	\begin{solution}
		Comenzamos definiendo al estimador suficiente como:
		\begin{tcolorbox}[colback=gray!15,colframe=black!1!black,title=Teorema 9.4 - Estimador suficiente]
			Let $U$ be a statistic based on the random sample $Y_1, Y_2,..., Y_n$. Then $U$ is a sufficient statistic for the estimation of a parameter $\theta$ if and only if the likelihood $L(\theta) = L(y_1, y_2, . . . , y_n | \theta)$ can be factored into two nonnegative functions,
			$$L(y_1,y_2,...,y_n |\theta)=g(u,\theta)\times h(y_1, y_2,..., y_n)$$
			where $g(u,\theta)$ is a function only of $u$ and $\theta$ and $h(y_1,y_2,...,y_n)$ is not a function of $\theta$.
		\end{tcolorbox}
	\begin{align*}
		L(y_1,y_2,\cdots, y_n|\mu) &= L(y_1|\mu)\times L(y_2|\mu)\times \cdots \times L(y_n|\mu)\\
		&=
			\frac{1}{\sigma\sqrt{2\pi}}\exp\left[{-(y_1-\mu)^2 \over (2\sigma^2)}\right]\times \frac{1}{\sigma\sqrt{2\pi}}\exp\left[{-(y_2-\mu)^2 \over (2\sigma^2)}\right]\times \cdots \times\\
			& \times \cdots \times\frac{1}{\sigma\sqrt{2\pi}}\exp\left[{-(y_n-\mu)^2 \over (2\sigma^2)}\right]\\
			&= \frac{1}{\sigma^n\left(\sqrt{2\pi}\right)^n}\times \exp\left[-\frac{1}{2\sigma^2}\left(\sum_{i=1}^{n}(y_i-\mu)^2\right)\right]\\
			&= \frac{1}{\sigma^n\left(\sqrt{2\pi}\right)^n}\times \exp\left[-\frac{1}{2\sigma^2}\sum_{i=1}^{n}\left(y^2_i-2y_i\mu+\mu^2\right)\right]\\
			&= \frac{1}{\sigma^n\left(\sqrt{2\pi}\right)^n}\times \exp\left[-\frac{1}{2\sigma^2}\left(\sum_{i=1}^{n}y^2_i-2\sum_{i=1}^{n}y_i\mu+\sum_{i=1}^{n}\mu^2\right)\right]\\
			&= \frac{1}{\sigma^n\left(\sqrt{2\pi}\right)^n}\times \exp\left[-\frac{1}{2\sigma^2}\left(\sum_{i=1}^{n}y^2_i-2n\overline{y}\mu+n\mu^2\right)\right]
			\intertext{Se conocía que $\sigma^2$=1, por lo cual:}
			&= \frac{1}{\left(\sqrt{2\pi}\right)^n}\times \exp\left[-\frac{1}{2}\left(\sum_{i=1}^{n}y^2_i-2n\overline{y}\mu+n\mu^2\right)\right]\\
			&= \frac{1}{\left(\sqrt{2\pi}\right)^n}\times \exp\left[-\frac{1}{2}\sum_{i=1}^{n}y^2_i+\frac{1}{2}2n\overline{y}\mu-\frac{1}{2}n\mu^2\right]\\
			&= \left\{\frac{1}{\left(\sqrt{2\pi}\right)^n}\times \exp\left[-\frac{1}{2}\sum_{i=1}^{n}y^2_i\right]\right\}\times\exp\left[n\overline{y}\mu-\frac{1}{2}n\mu^2\right]\\
			&= h(y)\times g(\overline{y},\mu)
			\end{align*}
		Por lo tanto, $\overline{y}$ es un estimador suficiente para $\mu$.
	\end{solution}
	\item b) ¿Cuál es la distribución de $\overline{Y}$ con sus parámetros? (Incluya la justificación).
	\begin{solution}
		Considérese el teorema 4.7:
		\begin{tcolorbox}[colback=gray!15,colframe=black!1!black,title=Teorema 7.1 ]
			Let $Y_1, Y_2, . . . , Y_n$ be a random sample of size n from a normal distribution 
			with mean $\mu$ and variance $\sigma$. Then
			$$\overline{Y}=\frac{1}{n}\sum_{i=1}^{n}Y_i$$
			is normally distributed with mean $\mu_{\overline{Y}} = \mu$ and variance $\sigma^2_{\overline{Y}} = \sigma^2/n$.
		\end{tcolorbox}
Entonces, podemos concluir que $\overline{Y}$ tiene una distribución normal con media $\mu_{\overline{Y}} = \mu$ y varianza $\sigma^2_{\overline{Y}} = 1/n$.
	\end{solution}
	\item c) Encuentre la función generadora de momentos de $\overline{Y}$ (Incluya la justificación).
	\begin{solution}
		Vamos a tomar como referencia el cuadro 2 del apéndice 2: 
		\begin{center}
			\includegraphics[scale=0.5]{/Users/rudiks/Git/Estadistica_Matematica/Parcial4/Images/Screen Shot 2021-05-23 at 20.39.19.png}
		\end{center}
	\end{solution}
	\item d) Calcule $E\left(\overline{Y}^{2}\right)$ y $E\left(\overline{Y}^{4}\right)$, utilizando la función generadora de momentos del inciso de
	$\overline{Y}$.
    \item e) Demuestre que el MUEV (estimador insesgado de varianza mínima) de $\mu^{2}$ es $\widehat{\mu^{2}}=$ $\overline{Y}^{2}-\frac{1}{n}$. \begin{solution}
    	content...
    \end{solution}
	\item f) Obtenga la $V A R\left(\widehat{\mu^{2}}\right)$, utilizando el resultado en $\mathrm{d}$ ).
	\begin{solution}
		content...
	\end{solution} 
\end{enumerate}
(Valor 25 puntos).


%\section{Problema 2}


Los estimadores de máxima verosimilitud tienen propiedades interesantes cuando se trabaja con muestras grandes. Suponga que $t(\theta)$ es una función derivable en $\theta$. Por la propiedad de invarianza, se tiene que si $\hat{\theta}$ es el MLE de $\theta$, entonces el MLE de $t(\theta)$ está dada por $t(\hat{\theta}) .$ En algunas condiciones de regularidad que se cumplen para las distribuciones que se consideren, $t(\hat{\theta})$ es un estimador consistente para $t(\theta) .$ Además, para tamaño de muestras grandes,
$$
Z=\frac{t(\hat{\theta})-t(\theta)}{\sqrt{\left[\frac{\partial t(\theta)}{\partial \theta}\right]^{2} \Big/ n E\left[-\frac{\partial^{2} \ln [f(Y \mid \theta)]}{\partial \theta^{2}}\right]}}
$$
tiene aproximandamente una distribución normal estándar, donde $f(Y \mid \theta)$ es la función densidad correspondiente a la distribución continua de interés, evaluada en el valor aleatorio Y. En el caso discreto, el resultado análogo se cumple con la función de probabilidad evaluada en el valor aleatorio $\mathrm{Y}$, $p(Y \mid \theta)$, se sustituye por la densidad $f(Y \mid \theta)$. En el caso particular, para una variable aleatoria con distribución de Bernoulli, $p(y \mid p)=$ $p^{y}(1-p)^{1-y}$, donde $\mathrm{y}=0,1 .$ Si $Y_{1}, \ldots, Y_{n}$ es una muestra aleatoria de tamaño $\mathrm{n}$ de dicha distribución.

\begin{enumerate}
	\item a) Encuentre el MLE para el parámetro p.
	\begin{solution}
		content...
	\end{solution}
	\item b) Encuentre el MLE para la expresión p(1-p).
	\item c) Construya un intervalo de confianza de $100(1-\alpha) \%$ para $\mathrm{p}(1-\mathrm{p})$, la varianza de dicha distribución (suponga una muestra grande).
\end{enumerate}
 (Valor 25 puntos).

%\section{Problema 3}

Suponga que $Y_{1}, Y_{2}, \ldots, Y_{n}$ es una muestra aleatoria de la distribución de Poisson con media $\lambda>0$.

\begin{enumerate}
	\item a) Encuentre el estimador del método de momentos para $\lambda$.
	\begin{solution}
		Comenzamos definiendo el estimador del método de momentos como:
		\begin{tcolorbox}[colback=gray!15,colframe=black!1!black,title=Method of moments]
			Choose as estimates those values of the parameters that are solutions of the equations $\mu'_k = m'_k$, for $k$ = 1,2,...,$t$, where $t$ is the number of parameters to be estimated. 
			$$\mu'_k=E(Y^k)\quad \text{ and } m'_k=\frac{1}{n}\sum_{i=1}^nY_i^k. $$
			\end{tcolorbox}
			Sabemos por hipótesis que $\mu$=$\lambda$. $\implies$ El estimador del método de momentos  está definido como: $$\hat{\lambda}=m'_1=\frac{1}{n}\sum_{i=1}^nY_i^1=\overline{Y}.$$
		
			\end{solution}
	
	\item b) Encuentre el estimador de máxima verosimilitud (MLE) $\hat{\lambda}$ para $\lambda$. 
	\begin{solution}
		Comenzamos definiendo el MLE, como: 
		\begin{tcolorbox}[colback=gray!15,colframe=black!1!black,title=Method of Maximum Likehood]
			Suppose that the likelihood function depends on k parameters $\theta_1,\theta_2,...,\theta_k$. Choose as estimates those values of the parameters that maximize the likelihood $L(y_1, y_2,..., y_n |\theta_1,\theta_2,...,\theta_k)$.
		\end{tcolorbox}
		
		\begin{tcolorbox}[colback=gray!15,colframe=black!1!black,title=Función de probabilidad de Poisson]
			Apéndice 2 - Distribuciones discretas, se define como: 
			$$p(y)=\frac{\lambda^ye^{-\lambda}}{y!}, \quad y=1,2,3,...$$
		\end{tcolorbox}
	\begin{align*}
		\intertext{$\implies$ Usando la definición de verosimilitud:}
		L(\lambda) &= p(y_1,y_2,\cdots, y_n | \lambda ) \\
								&= p(y_1|\lambda)\times  p(y_2|\lambda) \times \cdots  \times p(y_n|\lambda)\\
								&= \left\{\frac{\lambda^{y_1}e^{-\lambda}}{y_1!}\right\}\times \left\{\frac{\lambda^{y_2}e^{-\lambda}}{y_2!}\right\}\times \cdots \times \left\{\frac{\lambda^{y}e^{-\lambda}}{y!}\right\}\\
								&= \prod_{i=1}^{n}\frac{\lambda^{y_i}e^{-\lambda}}{y_i!}=\frac{\left(\prod_{i=1}^{n}\lambda^{y_i}\right)\left(\prod_{i=1}^{n}e^{-\lambda}\right)}{\prod_{i=1}^{n}y_i!}=\frac{\left(\lambda^{\sum_{i=1}^{n}y_i}\right)\left(ne^{-\lambda}\right)}{\prod_{i=1}^{n}y_i!}\\
		\ln\left[L(\lambda) \right]&=\ln \left[\frac{\left(\lambda^{\sum_{i=1}^{n}y_i}\right)\left(ne^{-\lambda}\right)}{\prod_{i=1}^{n}y_i!}\right]=\ln\left[\left(\lambda^{\sum_{i=1}^{n}y_i}\right)\left(ne^{-\lambda}\right)\right]-\ln\left[\prod_{i=1}^{n}y_i!\right]\\
		&=\ln\left[\left(\lambda^{\sum_{i=1}^{n}y_i}\right)\right]+\ln\left[\left(ne^{-\lambda}\right)\right]-\ln\left[\prod_{i=1}^{n}y_i!\right]\\
		&=\left[\sum_{i=1}^{n}y_i\right]\left[\ln(\lambda)\right]-\lambda n -\ln\left[\prod_{i=1}^{n}y_i!\right]\\
		\frac{d	\ln\left[L(\lambda) \right] }{d\lambda}&=\frac{1}{\lambda}\left[\sum_{i=1}^{n}y_i\right]-n-0
			\end{align*}
		Igualamos a 0 la expresión:
		$$\frac{1}{\hat{\lambda}}\left[\sum_{i=1}^{n}y_i\right]-n = 0\implies \hat{\lambda}=\frac{1}{n}\left[\sum_{i=1}^{n}y_i\right]=\overline{Y}$$
	\end{solution}
\end{enumerate}

(Valor 25 puntos).
\section{Problema 4}

Sea $Y$ una variable aleatoria que representa el número de éxitos en $\mathrm{n}$ intentos independientes con probabilidad p de éxito en cada intento. Además,
$$
Y=\sum_{i=1}^{n} Y_{i}
$$
donde
$$
Y_{i}=\begin{cases}
	1  , & \text { si el i-ésimo intento resulta en éxito } \\
	0  , & \text { en el otro caso }
\end{cases}
$$
para $\mathrm{i}=1, \ldots, \mathrm{n}$
\begin{enumerate}
	\item a) Demuestre que $\widehat{p_{n}}=\frac{Y}{n}$ es un estimador insesgado de $p$.
	\begin{solution} Tenemos las siguientes denificiones:: 
		\begin{tcolorbox}[colback=gray!15,colframe=black!1!black,title=Definición 8.2 - Sesgo ]
			Let $\hat{\theta}$ be a point estimator for a parameter $\theta$. Then $\hat{\theta}$  is an unbiased estimator if $E(\hat{\theta})=\theta$. If$E(\hat{\theta})\neq \theta$, $\theta$ is said to be biased.
			\end{tcolorbox}
		
		\begin{tcolorbox}[colback=gray!15,colframe=black!1!black,title=Definición 8.3 - Sesgo ]
			The bias of a point estimator $\hat{\theta}$ is given by $B(\hat{\theta}) = E(\hat{\theta}) - \theta $.
		\end{tcolorbox}
	A probar: $E(\hat{p}_n)=p$.  Entonces:

	$$E(\hat{p}_n)=E\left(\frac{Y}{n}\right)=\underbrace{\frac{1}{n}E(Y)}_{\text{Teorema 5.7}}=\frac{1}{n}E\left(\sum_{i=1}^{n}Y_i\right)=\frac{1}{n}E\left(Y_1+Y_2+\cdots +Y_n\right)=$$
	$$=\frac{1}{n}\left[\underbrace{E(Y_1)+E(Y_2)+\cdots + E(Y_n)}_{E(Y_i)=p \text{, definición de valor esperado.}}\right]=\frac{1}{n}\left[p+p+\cdots+p\right]=\frac{1}{n}(np)=p$$
	
	\end{solution}
	\item b) Demuestre que $\widehat{p_{n}}$ es un estimador consistente de $p$.
	\begin{solution} 
		Procedemos a calcular la varianza del estimador, es decir: 
		$$VAR(\hat{p}_n)=\frac{pq}{n},\quad q=(1-p). \quad \text{ (Deducción en el ejercicio 5.28).}$$
		$\implies$ Tomamos como referencia el teorema 9.1:
		\begin{tcolorbox}[colback=gray!15,colframe=black!1!black,title=Teorema 9.1]
			An unbiased estimator $\hat{\theta}_n$ for $\theta$ is a consistent estimator of $\theta$ if
			$$\lim_{n\to\infty}VAR(\hat{\theta}_n)=0.$$
		\end{tcolorbox}
	$$\implies \lim_{n\to\infty}VAR(\hat{p}_n)=\lim_{n\to\infty}VAR\left(\frac{pq}{n}\right)=VAR\left(0\right)=0.$$
	\end{solution}
	\item c) Cuando $\mathrm{n}$ es grande, demuestre que la distribución de $\frac{\widehat{p_{n}}-p}{\sqrt{p(1-p) / n}}$ converge a una distribución normal estándar.
	\begin{solution}
		Considerando el teorema del límite central:
		\begin{tcolorbox}[colback=gray!15,colframe=black!1!black,title=Teorema 7.4]
			Let $Y_1, Y_2, . . . , Y_n$ be independent and identically distributed random variables with $E (Y_i ) = \mu$ and $V (Y_i ) = \sigma^ 2 < \infty$. Define
			$$U_n=\frac{\sum_{i=1}^{n}Y_i-n\mu }{\sigma\sqrt{n}}=\frac{\overline{Y}-\mu}{\sigma/\sqrt{n}}.$$
			Then, the distribution function $U_n$ converges to the standard normal distribution function as $n\to\infty$. 
		\end{tcolorbox}
	Dados los 2 incisos anteriores tenemos $E(Y_i)=p$ y $VAR(Y_i)=p(1-p)$. Por hipótesis, sabemos $Y=\sum_{i=1}^{n}Y_i$. Definimos: 
	$$U_n=\frac{Y-np}{\sqrt{p(1-p)}\sqrt{n}}=\frac{\frac{Y-np}{n}}{\frac{\sqrt{p(1-p)}\sqrt{n}}{n}}=\frac{\hat{p}_n-p}{\sqrt{\frac{p(1-p)}{n}}}$$
	Por lo tanto, la distribución converge a una distribución normal.
	\end{solution}
	\item d) Cuando $\mathrm{n}$ es grande, demuestre que la distribución de $\frac{\widehat{p_{n}}-p}{\sqrt{\widehat{p_{n}}\left(1-\widehat{p_{n}}\right) / n}}$ converge a una distribución normal estándar. 
	\begin{solution}
	Sabemos que $\hat{p}_n$ es consistente, por lo que $(1-\hat{p}_n)$ también debe ser consistente; por el inciso $b$ del teorema 9.2, entonces $\hat{p}_n(1-\hat{p}_n)$ es consiste para $p(1-p)$. 
	$$\implies \frac{U_n}{W_n}=\frac{\hat{p}_n-p}{\sqrt{\frac{\hat{p}_n(1-\hat{p}_n)}{n}}}=\frac{\frac{\hat{p}_n-p}{\sqrt{\frac{p(1-p)}{n}}}}{\frac{\sqrt{\frac{\hat{p}_n(1-\hat{p}_n)}{n}}}{\sqrt{\frac{p(1-p)}{n}}}}=\frac{\underbrace{\frac{\hat{p}_n-p}{\sqrt{\frac{p(1-p)}{n}}}}_{\text{Inciso anterior.}}}{\underbrace{\sqrt{\frac{\hat{p}_n(1-\hat{p}_n)}{p(1-p)}}}_{\text{Su probabilidad converge a 1.}}}$$
	Por lo tanto, $U_n$ converge a una distribución normal y la probabilidad  de $W_n$  converge a 1. Considerando: \begin{tcolorbox}[colback=gray!15,colframe=black!1!black,title=Teorema 9.3]
		Suppose that $U_n$ has a distribution function that converges to a standard normal distribution function as $n \to \infty$. If $W_n$ converges in probability to 1, then the distribution function of $U_n / W_n $converges to a standard normal distribution function.
	\end{tcolorbox}

Se concluye que la distribución $U_n/W_n$ converge a una distribución normal estándar.
 	\end{solution}
\end{enumerate}
(Valor 25 puntos)

%---------------------------
\bibliographystyle{apalike}
\bibliography{sample.bib}

\end{document}